\usepackage{cmap}
\usepackage[T1]{fontenc} 
\usepackage[utf8]{inputenc}
\usepackage[english,russian]{babel}

\usepackage{colortbl}

\usepackage{caption}
\usepackage{subcaption}
\usepackage{textcomp}
\usepackage{enumitem}

\usepackage{float}



\usepackage{enumitem}

\usepackage{graphicx}
\usepackage{multirow}


\usepackage{pgfplots}
\pgfplotsset{compat=newest}
\usepgfplotslibrary{units}

\usepackage{caption}
\captionsetup{labelsep=endash}
\captionsetup[figure]{name={Рисунок}}
\captionsetup[subtable]{labelformat=simple}
\captionsetup[subfigure]{labelformat=simple}
\renewcommand{\thesubtable}{\text{Таблица }\arabic{chapter}\text{.}\arabic{table}\text{.}\arabic{subtable}\text{ --}}
\renewcommand{\thesubfigure}{\text{Рисунок }\arabic{chapter}\text{.}\arabic{figure}\text{.}\arabic{subfigure}\text{ --}}


\usepackage{textcomp}

\usepackage{amsmath}
\usepackage{amsfonts}
\usepackage{array}

\usepackage{geometry}
\geometry{left=30mm}
\geometry{right=15mm}
\geometry{top=20mm}
\geometry{bottom=20mm}
\geometry{foot=1.7cm}

\usepackage{titlesec}
\titleformat{\section}
{\normalsize\bfseries}
{\thesection}
{1em}{}
\titlespacing*{\chapter}{0pt}{-30pt}{8pt}
\titlespacing*{\section}{\parindent}{*4}{*4}
\titlespacing*{\subsection}{\parindent}{*4}{*4}

\usepackage{enumitem}
\usepackage{xcolor}
\usepackage{listings}
\usepackage{multicol}
\usepackage{xpatch}
\usepackage{realboxes}



% Маркировка для списков
\def\labelitemi{$\circ$}
\def\labelitemii{$*$}


\usepackage{setspace}
\onehalfspacing % Полуторный интервал

\frenchspacing
\usepackage{indentfirst} % Красная строка

\usepackage{titlesec}
\usepackage{xcolor}
% Названия глав
\titleformat{\section}{\normalsize\textmd}{\thesection}{1em}{}

\definecolor{gray35}{gray}{0.35}

\newcommand{\hsp}{\hspace{20pt}} % длина линии в 20pt

\titleformat{\chapter}[hang]{\huge}{\textcolor{gray35}{\thechapter.}\hsp}{0pt}{\huge\scshape}

\titleformat{\section}{\Large}{\textcolor{gray35}\thesection}{20pt}{\Large\scshape}
\titleformat{\subsection}{\Large}{\thesubsection}{20pt}{\Huge\textmd}
\titleformat{\subsubsection}{\normalfont\textmd}{}{0pt}{}

% Настройки введения

\addtocontents{toc}{\setcounter{tocdepth}{2}}
\addtocontents{toc}{\setcounter{secnumdepth}{1}}

\usepackage{tocloft,lipsum,pgffor}

\addtocontents{toc}{~\hfill\textnormal{Страница}\par}

\renewcommand{\cftpartfont}{\normalfont\textmd}

\addto\captionsrussian{\renewcommand{\contentsname}{Содержание}}
\renewcommand{\cfttoctitlefont}{\Huge\textmd}

\renewcommand{\cftchapfont}{\normalfont\normalsize}
\renewcommand{\cftsecfont}{\normalfont\normalsize}
\renewcommand{\cftsubsecfont}{\normalfont\normalsize}
\renewcommand{\cftsubsubsecfont}{\normalfont\normalsize}

\renewcommand{\cftchapleader}{\cftdotfill{\cftdotsep}}

\usepackage{listings}
\usepackage{xcolor}
%\usepackage{algorithm}
%\usepackage{algpseudocode}

\lstnewenvironment{itemlisting}[1][]
{%
	\mbox{}
	\vspace*{-\baselineskip}
	\lstset{
		xleftmargin=\leftmargin,
		linewidth=\linewidth,
		#1
	}%
}
{}

\newcommand{\olsi}[1]{\,\overline{\!{#1}}}

\bibliographystyle{utf8gost705u.bst}
\usepackage[backend=biber,
sorting=none,
]{biblatex} 

\usepackage{tasks}

\addbibresource{ref-lib.bib} % База библиографии

\usepackage[pdftex]{hyperref} % Гиперссылки
\hypersetup{hidelinks}

% Листинги 
\usepackage{listings}
\lstset{
	literate=
	{а}{{\selectfont\char224}}1
	{б}{{\selectfont\char225}}1
	{в}{{\selectfont\char226}}1
	{г}{{\selectfont\char227}}1
	{д}{{\selectfont\char228}}1
	{е}{{\selectfont\char229}}1
	{ё}{{\"e}}1
	{ж}{{\selectfont\char230}}1
	{з}{{\selectfont\char231}}1
	{и}{{\selectfont\char232}}1
	{й}{{\selectfont\char233}}1
	{к}{{\selectfont\char234}}1
	{л}{{\selectfont\char235}}1
	{м}{{\selectfont\char236}}1
	{н}{{\selectfont\char237}}1
	{о}{{\selectfont\char238}}1
	{п}{{\selectfont\char239}}1
	{р}{{\selectfont\char240}}1
	{с}{{\selectfont\char241}}1
	{т}{{\selectfont\char242}}1
	{у}{{\selectfont\char243}}1
	{ф}{{\selectfont\char244}}1
	{х}{{\selectfont\char245}}1
	{ц}{{\selectfont\char246}}1
	{ч}{{\selectfont\char247}}1
	{ш}{{\selectfont\char248}}1
	{щ}{{\selectfont\char249}}1
	{ъ}{{\selectfont\char250}}1
	{ы}{{\selectfont\char251}}1
	{ь}{{\selectfont\char252}}1
	{э}{{\selectfont\char253}}1
	{ю}{{\selectfont\char254}}1
	{я}{{\selectfont\char255}}1
	{А}{{\selectfont\char192}}1
	{Б}{{\selectfont\char193}}1
	{В}{{\selectfont\char194}}1
	{Г}{{\selectfont\char195}}1
	{Д}{{\selectfont\char196}}1
	{Е}{{\selectfont\char197}}1
	{Ё}{{\"E}}1
	{Ж}{{\selectfont\char198}}1
	{З}{{\selectfont\char199}}1
	{И}{{\selectfont\char200}}1
	{Й}{{\selectfont\char201}}1
	{К}{{\selectfont\char202}}1
	{Л}{{\selectfont\char203}}1
	{М}{{\selectfont\char204}}1
	{Н}{{\selectfont\char205}}1
	{О}{{\selectfont\char206}}1
	{П}{{\selectfont\char207}}1
	{Р}{{\selectfont\char208}}1
	{С}{{\selectfont\char209}}1
	{Т}{{\selectfont\char210}}1
	{У}{{\selectfont\char211}}1
	{Ф}{{\selectfont\char212}}1
	{Х}{{\selectfont\char213}}1
	{Ц}{{\selectfont\char214}}1
	{Ч}{{\selectfont\char215}}1
	{Ш}{{\selectfont\char216}}1
	{Щ}{{\selectfont\char217}}1
	{Ъ}{{\selectfont\char218}}1
	{Ы}{{\selectfont\char219}}1
	{Ь}{{\selectfont\char220}}1
	{Э}{{\selectfont\char221}}1
	{Ю}{{\selectfont\char222}}1
	{Я}{{\selectfont\char223}}1
}
\lstset{
	language=Matlab,
	morekeywords={matlab2tikz},
	numbers=left, 
	numberstyle=\tiny,
	stepnumber=1,
	firstnumber=1,
	numbersep=5pt,
	frame=single,
	basicstyle=\footnotesize, 
	showstringspaces=false,
	breaklines=true,
}



\definecolor{darkgray}{gray}{0.15}

\definecolor{teal}{rgb}{0.25,0.88,0.73}
\definecolor{gray}{rgb}{0.5,0.5,0.5}
\definecolor{b-red}{rgb}{0.88,0.25,0.41}
\definecolor{royal-blue}{rgb}{0.25,0.41,0.88}


% какой то сложный кусок со стак эксчейндж для квадратных скобок
\makeatletter
\newenvironment{sqcases}{%
	\matrix@check\sqcases\env@sqcases
}{%
	\endarray\right.%
}
\def\env@sqcases{%
	\let\@ifnextchar\new@ifnextchar
	\left\lbrack
	\def\arraystretch{1.2}%
	\array{@{}l@{\quad}l@{}}%
}
\makeatother

% и для матриц
\makeatletter
\renewcommand*\env@matrix[1][\arraystretch]{%
	\edef\arraystretch{#1}%
	\hskip -\arraycolsep
	\let\@ifnextchar\new@ifnextchar
	\array{*\c@MaxMatrixCols c}}
\makeatother
